\documentclass{amsart}

\newcommand{\R}{\mathbb{R}}
\newcommand{\dom}{\textrm{dom}}
\newcommand{\dist}{\textrm{dist}}
\newcommand{\calU}{\mathcal{U}}
\newcommand{\calO}{\mathcal{O}}

\usepackage{amssymb, amsmath}
\usepackage{enumerate}

\usepackage{palatino}

\newtheorem{assumption}{Assumption}
\newtheorem{theorem}{Theorem}
\newtheorem{lemma}{Lemma}
\newtheorem{corollary}{Corollary}

\title{Spectral Perturbation Theory}
\author{Notes by Parker Knight}


\begin{document}
\maketitle

\section{Introduction}

This set of notes will closely follow the second chapter of \cite{chen_spectral_2021}.

The general problem is defined as follows: suppose there exists some matrix of interest $M^{*}$, and we observe a perturbed version $M = M^{*} + E$ where $E$ is a perturbation matrix. We are interested in characterizing how the spectral properties of $M^{*}$ (i.e. the eigenspace) change in light of the perturbation $E$. To do so, we will describe the classic Davis-Kahan $\sin \Theta$ theorem for symmetric matrices. But first, we describe various metrics for describing the distance between subspaces.

Throughout the note set, we let $\calU^{*}$ and $\calU$ denote two $r$ dimensional subspaces in $\R^{n}$. Let $U^{*}$ and $U$ be matrices in $\R^{n \times r}$ whose columns form an orthonormal basis for $\calU^{*}$ and $\calU$ respectively. For any matrix $A$, let $A_{\perp}$ denote its orthogonal compliment. Let $\calO^{r \times r}$ denote the set of orthogonal matrices in $\R^{r \times r}$, and let $||.||_{op}$ denote the operator norm (largest singular value) of a matrix.

\section{Distance between subspaces}

A key challenge in describing the distance between subspaces is the notion of rotational ambiguity, namely for for any rotation matrix $R \in R^{r \times r}$, we have that $UR$ is also an orthogonal basis for $\calU$. So, even when $\calU^{*} = \calU$, we may have $|||U - U^*||| \neq 0$ for our matrix norm of choice $||| . |||$, depending on how the bases are rotated. Any useful distance metric on subspaces much account for this rotational ambiguity. Following the approach of \cite{chen_spectral_2021}, we describe a few difference useful choices of metric.

\subsection{Distance with optimal rotation}

A natural approach to addressing the rotational invariance problem is to simply choose the rotation of $U$ which is
closest in norm to $U^{*}$. This yields the following distance metric:

$$\dist(U, U^{*}) := \min_{R \in \calO^{r \times r}}|||UR - U^{*}|||$$

\subsection{Distance between projections}

Recall that the projection onto $\calU$ is given by $UU^{T}$. A useful fact is that this projection matrix is unchanged by its rotation: for any $R \in \calO^{r \times r}$, we have $UR(UR)^{T} = URR^{T}U^{T} = UU^{T}$. This motivates the following metric between subspaces:

$$\dist_{p}(U, U^{*}) := |||UU^{T} - U^{*}U^{*T}|||$$

\subsection{Distance via principal angles}

Let $\sigma_{1} \geq \sigma_{2} \geq ... \geq \sigma_{r} \geq 0$ be the singular values of $U^{T}U^{*}$ arranged in descending order. By the properties of the operator norm\footnote{This follows from the definition: Recall for any $A$ and $x$ with agreeing dimension, we have $||Ax||_2 \leq ||A||_{op}||x||_{2}$. It follows that $||BAx||_{2} \leq ||B||_{op}||Ax||_{2} \leq ||B||_{op}||A||_{op}||x||_{2}$. Thus, $||BA||_{op} \leq ||B||_{op}||A||_{op}$.}, we know that $||U^TU^*||_{op} \leq ||U||_{op} ||U^*||_{op} = 1$. So, it follows that the singular values all fall with $[0.1]$. Using this fact, we define the principal angles between $\calU$ and $\calU^{*}$ as

$$\theta_{i} := \arccos(\sigma_{i})$$

for $i = 1, ..., r$. These principal angles satisfy

$$0 \leq \theta_{1} \leq \theta_{r} \leq \pi/2$$

Join these angles into a diagonal matrix $\Theta$, and define the following metric

$$\dist_{\sin}(U, U^{*}) := |||\sin \Theta|||$$

where $\sin$ is applied elementwise to $\Theta$, and $|||.|||$ is a matrix norm of choice.

\bigskip

It can be shown (ommitted here) that these three distance metrics are equivalent up to a scaling factor of $\sqrt{2}$ when the norm of choice is the operator norm or the Frobenius norm. Following the example of \cite{chen_spectral_2021}, we will use the optimal rotation metric throughout these notes.

\section{Davis-Kahan}

Now we are able to develop a theory for understanding the effect of perturbation on eigenspaces.

\bigskip

Throughout this section, let $M^{*}$ and $M = M^{*} + E$ be $n \times n$ real symmetric matrices, with eigendecompositions

\begin{align*}
  M^{*} &= \begin{bmatrix}U^{*} & U^{*}_{\perp} \end{bmatrix} \begin{bmatrix}\Lambda^{*} & 0 \\ 0 & \Lambda^{*}_{\perp} \end{bmatrix} \begin{bmatrix} U^{*T} \\ U^{*T}_{\perp}\end{bmatrix} \\
  M &= \begin{bmatrix}U & U_{\perp} \end{bmatrix} \begin{bmatrix}\Lambda & 0 \\ 0 & \Lambda_{\perp} \end{bmatrix} \begin{bmatrix} U^{T} \\ U^{T}_{\perp}\end{bmatrix} \\
\end{align*}

where $U^{*}$ and $U$ lie in $\R^{n \times r}$. Now we present the Davis-Kahan theorem.

\begin{theorem}
  Suppose that
  \begin{align*}
    &\textrm{diag} (\Lambda^{*}) \subseteq [\alpha, \beta] \\
    &\textrm{diag} (\Lambda_{{\perp}}) \subseteq (-\infty, \alpha - \Delta] \cup [\beta + \Delta, \infty)
  \end{align*}

  for some $\alpha, \beta \in \R$ and $\Delta > 0$. Then:

  \begin{align*}
    &\dist_{op} (U, U^{*}) \leq \sqrt{2}||\sin \Theta ||_{op} \leq \frac{\sqrt{2}||EU^{*}||_{op}}{\Delta} \leq \frac{\sqrt{2}||E||_{op}}{\Delta} \\
    &\dist_{F}(U, U^{*}) \leq \sqrt{2}||\sin \Theta ||_{F} \leq \frac{\sqrt{2}||EU^{*}||_{F}}{\Delta} \leq \frac{\sqrt{2}||E||_{op}}{\Delta}
   \end{align*}

\end{theorem}

\begin{proof}

  We prove the theorem by controlling the distance $|||U^{T}_{\perp}U^{*}|||$ where $|||.|||$ is unitarily invariant.

  Without loss of generality, we constrain ourselves to the case where $\alpha = -\beta \leq 0$\footnote{If case is violated, we can 'center' the matrices $M$ and $M^{*}$ and recover the same setting; see \cite{chen_spectral_2021}.}. This assumption, combined with the assumption of the theorem, give us the following:

  $$||\Lambda^{*}||_{op} \leq \beta$$

  $$\sigma_{\min}(\Lambda^{\perp}) \geq \beta + \Delta$$

  Now we can study $U^{T}_{\perp}U^{*}$. Using the properties of eigenvectors, we first observe the following identity:

  $$U^{T}_{\perp}EU^{*} = U^{T}_{\perp}(M - M^{*})U^{*} = \Lambda_{\perp}U^{T}_{\perp}U^{*} - U_{\perp}^{T}U^{*}\Lambda^{*}$$

  Letting $R := EU^{*}$, we apply the triangle inequality to obtain:

 \begin{align*}
   |||U^{T}_{\perp}R||| &\geq |||\Lambda_{\perp}U^{T}_{\perp}U^{*}||| - |||U_{\perp}^{T}U^{*}\Lambda^{*}||| \\
                        &\geq \sigma_{\min}(\Lambda_{\perp})|||U_{\perp}^{T}U^{*}||| - ||\Lambda^{*}||_{op}|||U_{\perp}^{T}U^{*}||| \\
  &\geq (\beta + \Delta - \beta) |||U^{T}_{\perp}U^{*}||| = \Delta |||U^{T}_{\perp}U^{*}|||
 \end{align*}

 It immediately follows that

 $$|||U^{T}_{\perp}U^{*}||| \leq \frac{|||U_{\perp}^{T}U^{*}}{\Delta} \leq \frac{|||R|||}{\Delta} = \frac{|||EU^{*}|||}{\Delta}$$

 and we are done.
\end{proof}


\bibliography{refs}
\bibliographystyle{ieeetr}

\end{document}
