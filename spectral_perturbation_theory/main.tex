\documentclass{amsart}

\newcommand{\R}{\mathbb{R}}
\newcommand{\dom}{\textrm{dom}}
\newcommand{\dist}{\textrm{dist}}
\newcommand{\calU}{\mathcal{U}}
\newcommand{\calO}{\mathcal{O}}

\usepackage{amssymb, amsmath}
\usepackage{enumerate}

\usepackage{palatino}

\newtheorem{assumption}{Assumption}
\newtheorem{theorem}{Theorem}
\newtheorem{lemma}{Lemma}
\newtheorem{corollary}{Corollary}

\title{Spectral Perturbation Theory}
\author{Notes by Parker Knight}


\begin{document}
\maketitle

\section{Introduction}

This set of notes will closely follow the second chapter of \cite{chen_spectral_2021}. My aim is to describe some useful matrix perturbation theorems, give their proofs, and describe an application of the theory to a problem in statistics.

The general problem is defined as follows: suppose there exists some matrix of interest $M^{*}$, and we observe a perturbed version $M = M^{*} + E$ where $E$ is a perturbation matrix. We are interested in characterizing how the spectral properties of $M^{*}$ (i.e. the eigenspace or singular subspace) change in light of the perturbation $E$. To do so, we will describe the classic Davis-Kahan $\sin \Theta$ theorem for symmetric matrices, and Wedin's extension to general matrices. But first, we describe various metrics for describing the distance between subspaces.

Throughout the note set, we let $\calU^{*}$ and $\calU$ denote two $r$ dimensional subspaces in $\R^{n}$. Let $U^{*}$ and $U$ be matrices in $\R^{n \times r}$ whose columns form an orthonormal basis for $\calU^{*}$ and $\calU$ respectively. For any matrix $A$, let $A_{\perp}$ denote its orthogonal compliment. Let $\calO^{r \times r}$ denote the set of orthogonal matrices in $\R^{r \times r}$.

\section{Distance between subspaces}

A key challenge in describing the distance between subspaces is the notion of rotational ambiguity, namely for for any rotation matrix $R \in R^{r \times r}$, we have that $UR$ is also an orthogonal basis for $\calU$. So, even when $\calU^{*} = \calU$, we may have $|||U - U^*||| \neq 0$ for our matrix norm of choice $||| . |||$, depending on how the bases are rotated. Any useful distance metric on subspaces much account for this rotational ambiguity. Following the approach of \cite{chen_spectral_2021}, we describe a few difference useful choices of metric.

\subsection{Distance with optimal rotation}

A natural approach to addressing the rotational invariance problem is to simply choose the rotation of $U$ which is
closest in norm to $U^{*}$. This yields the following distance metric:

$$\dist(U, U^{*}) := \min_{R \in \calO^{r \times r}}|||UR - U^{*}|||$$

\subsection{Distance between projections}

Recall that the projection onto $\calU$ is given by $UU^{T}$. A useful fact is that this projection matrix is unchanged by its rotation: for any $R \in \calO^{r \times r}$, we have $UR(UR)^{T} = URR^{T}U^{T} = UU^{T}$. This motivates the following metric between subspaces:

$$\dist_{p}(U, U^{*}) := |||UU^{T} - U^{*}U^{*T}|||$$

\subsection{Distance via principal angles}


\section{Davis-Kahan}

\section{Wedin}

\section{An application}


\bibliography{refs}
\bibliographystyle{ieeetr}

\end{document}
