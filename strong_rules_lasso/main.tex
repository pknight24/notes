\documentclass{amsart}

\newcommand{\R}{\mathbb{R}}
\newcommand{\dom}{\textrm{dom}}

\usepackage{amssymb, amsmath}

\title{Strong Rules for Efficient LASSO Computations and the BASIL Algorithm}
\author{Notes by Parker Knight}


\begin{document}

\maketitle

\section{Strong rules for the LASSO}

\subsection{Subgradients}

Let $f: \R^m \rightarrow \R$ be convex.

\bigskip

Recall the following first order condition: if $f$ is differentiable, then 

$$f(y) \geq f(x) + \nabla f(x)^T(y - x) \quad \forall x, y \in \dom(f)$$

What if $f$ is not differentiable? This motivates the following definition: call
$g \in \R^m$ a \textit{subgradient} of $f$ at $x$ iff

$$f(y) \geq f(x) + g^T(y - x) \quad \forall x, y \in \dom(f)$$

The subdifferential of $f$ at $x$, denote $\partial f(x)$ is the set of all
subgradients. The following facts will be useful:

\begin{enumerate}
	\item If $f$ is differentiable, then $\partial f(x) = \{\nabla f(x) \}$
	\item For $\alpha_1, \alpha_2 \geq 0$, then $\partial \left[\partial
	\alpha_1 f_1(x) + \alpha_2 f_2(x) \right] = \alpha_1 \partial f_1(x) +
	\alpha_2 \partial f_2(x)$
	\item $x^*$ minimizes $f$ iff $0 \in \partial f(x^*)$
\end{enumerate}

where we define set addition as $A + B = \{a + b | a \in A, b \in B \}$. 
\subsection{The LASSO}

\subsection{Strong rules}

\section{BASIL}


\bibliography{refs}
\bibliographystyle{ieeetr}
\end{document}